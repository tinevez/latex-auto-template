\leadauthor{Tinevez}

\title{An automated GitHub template for LaTeX articles.}

\shorttitle{GitHub auto-LaTeX}

\author[1\space\Letter]{Jean-Yves Tinevez}

\affil[1]{Image Analysis Hub, C2RT / DTPS, Institut Pasteur, Paris, France}  

\maketitle

\begin{abstract} % abstract

LaTeX is a really nice document processing tool that can produce beautiful articles, preprints, manuals, reports, \textit{etc}.
Here we introduced a GitHub template repository that automated the compilation and deployment of the document produced by LaTeX in this repository, thanks to GitHub actions.
\end{abstract}


\begin{keywords}
    Writing | LaTeX | GitHub | GitHub Actions | Continuous Integration | Continuous Deployment
\end{keywords}


\begin{corrauthor}
    jean-yves.tinevez\at pasteur.fr
\end{corrauthor}



\section*{Introduction}

Despite its old age, LaTeX is still the favourite document processing of many of us, even in Biology-related fields.
Because it relies on plain text files to produce the document, it fits very well with the source control tools and continuous integration practices that are found in software development.
Several platforms offer source code hosting with version control, and include collaboration tools and continuous integration facilities. 
GitHub is such a platform, and is used frequently for academic open-source softwares
