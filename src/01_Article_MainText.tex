\leadauthor{Hernriques and Tinevez}

\title{A GitHub template for the automated production of preprints and LaTeX articles.}

\shorttitle{GitHub auto-LaTeX}

\author[1\space\Letter]{Ricardo Henriques}
\author[2\space\Letter]{Jean-Yves Tinevez}

\affil[1]{Optical Cell Biology unit, Instituto Gulbenkian de Ciência, Oeiras, Portugal}
\affil[2]{Image Analysis Hub, C2RT / DTPS, Institut Pasteur, Paris, France}  

\maketitle

\begin{abstract} % abstract

LaTeX is a really nice document processing tool that can produce beautiful articles, preprints, manuals, reports, \textit{etc}.
Here we introduced a GitHub template repository that automated the compilation and deployment of the document produced by LaTeX in this repository, thanks to GitHub actions.
\end{abstract}


\begin{keywords}
    Writing | LaTeX | GitHub | Overleaf | GitHub Actions | Continuous Integration | Continuous Deployment
\end{keywords}


\begin{corrauthor}
    jean-yves.tinevez\at pasteur.fr
\end{corrauthor}



\section*{Introduction}

Since the advent of the bioRxiv hosting platform in 2013, the production of preprints became a common practice in Biology-related fields.
This initiative fostered what was first a cautious interests from these fields~\cite{Callaway2012}.
The bioRxiv platform quickly entered exponential growth soon after its launch~\cite{Callaway2013} and in 2018, there have been more than 2000 preprints uploaded each month~\cite{Abdill2019}.
Nowadays preprints in Biology seem to be used more to establish precedence~\cite{Vale2016} and facilitate dissemination than to foster collaborative reviews~\cite{Anderson2020}.
They also have been shown to have a positive impact on citation and exposure~\cite{Abdill2019, Fraser2019}.

The production of documents to be published as preprints has become therefore a common task of many researchers.
While unedited, unformatted word documents should suffice, a more visually pleasing document will ease the reading of the preprint and be subjectively better valued by its readership.
But this requires a supplemental effort from the author, which is classically handled by a journal edition team after article acceptance.

Despite its old age, LaTeX~\cite{Lamport1994} is still a favourite document processing of many scientists, even in Biology-related fields.
It can typeset content in documents with a quality that matches the production of professional editing teams.
One of us recently created a document class for LaTeX specifically made for scientific publications~\cite{HenriquesPreprintTemplate}.
LaTeX offers others advantages.
Because it relies on plain text files to produce the document, it fits very well with the version control tools and continuous integration practices that are found in software development.
Several platforms offer source code hosting with version control~\cite{SourceForge, Mercurial, GitHub, Bitbucket, GitLab}, and include collaboration  and continuous integration facilities. 
Continuous integration \TODO

GitHub~\cite{GitHub} is such a platform, and is used frequently for the development of academic open-source software, for instance in the bioimage analysis domain~\cite{FijiPaper, IcyPaper, ImgLib2Paper}.






\section*{Bibliography}
\bibliographystyle{zHenriquesLab-StyleBib}
\bibliography{Bibliography}
